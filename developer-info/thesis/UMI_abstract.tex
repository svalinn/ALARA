\documentclass[12pt,dvips]{article}

\usepackage{vmargin}
\setpapersize{USletter}
\setmarginsrb{1.25in}{1in}{1.1in}{1.1in}{0.25in}{0.25in}{0in}{0in}
\def\ALARA{\textsf{ALARA}}
\renewcommand\baselinestretch{1.66}
\pagestyle{empty}

\begin{document}

\begin{center}
  \textbf{\large ALARA: Analytic and Laplactian Adaptive Radioactivity
    Analysis}\\
  \ \\
  \textbf{Paul Philip Hood Wilson}\\
  Under the supervision of Associate Professor Douglass L.~Henderson\\
  At the University of Wisconsin-Madison
\end{center}


  While many codes have been written to compute the induced activation
  and changes in composition caused by neutron irradiation, most of
  those which are still being updated are only slowly adding
  functionality and not improving the accuracy, speed and usability of
  their existing methods.  \ALARA\ moves forward in all four of these
  areas, with primary importance being placed on the accuracy and
  speed of solution.
  
  By carefully analyzing the various ways to model the physical
  system, the methods to solve the mathematical problem and the
  interaction between these two issues, \ALARA\ chooses an optimum
  combination to achieve high accuracy, fast computation, and enhanced
  versatility and ease of use.  In addition to a set of base features,
  standard to any activation code, \ALARA\ offers a number of
  extensions, including arbitray hierarchical irradiation schedules
  and a form of reverse problem for calculating the detailed
  activation of specific isotopes.
  
  The physical system is modeled using advanced linear chains, which
  include the contributions from straightened loops in the reaction
  scheme, while the truncation philosophy minimizes the discrepancies
  between the model and the real problem.  The mathematical method is
  then adaptively chosen based on the characteristics of each linear
  chain to use analytically exact methods when possible and an
  accurate expansion technique otherwise.
  
  \ALARA\ has been successfully validated against established fusion
  activation codes using a standard activation benchmark problem.  In
  addition to demonstrating \ALARA's accuracy, this validation
  excerise has demonstrated its speed.  Furthermore, by extending the
  benchmark problem to validate its advanced features, \ALARA's
  flexibility has been proven.
  
  With its modern computational techniques and continuing development,
  it is hoped that \ALARA\ will become a widely used code for the
  activation analysis of nuclear systems.
\end{document}

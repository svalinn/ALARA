
\begin{chapter}{Introduction and Background}
  
  Since the advent of the nuclear age nearly six decades ago, it has
  become necessary to study and simulate the effect of radiation on
  materials.  Most nuclear systems, including existing fission power
  reactors, anticipated fusion power reactors, and a wide variety of
  experimental facilities, produce large numbers of energetic
  neutrons.  These neutrons interact with the systems' materials,
  inducing a variety of responses.

  \begin{section}{Problem Definition\label{sec:intro.prob_def}}
    
    \begin{floatingfigure}{0.55\columnwidth}
      \begin{center}
        \epsfig{file=eps/basic_tree.eps,width=0.5\columnwidth}
        \caption{A sample activation tree showing the results of
          activation of isotope A.}\label{fig:intro.basic_tree}
      \end{center}
    \end{floatingfigure}
    
    \noindent Activation is just one of the many possible responses resulting
    from the neutron irradiation of materials.  The neutrons interact
    with the material's nuclei, converting them to different isotopes.
    With many such reactions possible for most isotopes, each of the
    original material's isotopes can be partially converted into over
    20 others after just one generation.  These isotopes, in turn, can
    undergo similar interactions, leading to yet more isotopes, and so
    on.  Furthermore, many of these isotopes may be radioactive, and
    their decay products introduce even more isotopes to the physical
    system.  If represented graphically (Figure
    \ref{fig:intro.basic_tree}), this process forms a tree of
    isotopes, where each branch in the tree represents either a
    nuclear reaction (\textsl{e.g.} $A \rightarrow B$) or a nuclear
    decay (\textsl{e.g.} $C \rightarrow G$).  This process of
    converting a non-radioactive material to a radioactive one is
    known as activation.
    
    After calculating the concentrations of all the various isotopes
    created by the activation process, other engineering responses can
    be determined.  Multiplying by the radioactive isotopes' decay
    constants ($\lambda = \ln 2/t_{\frac{1}{2}}$) determines the
    material's radioactivity.  Further multiplying by the average
    energy of each decay, these radioactivity values can be converted
    to decay heat results.  By comparing the radioactivity to
    regulated limits, the waste disposal ratings can be determined,
    indicating how the material must be handled.  Incorporating the
    gamma ray emission information for each radioactive isotopes gives
    the gamma ray source, which may be used as the source term for a
    radiation transport calculation to determine a radiation dose at
    some spatial point.
    
    These responses are essential when designing, operating and
    costing a nuclear system.  For safety considerations, it is
    important to know the inventory of all the radioactive isotopes
    which may be released and to know the decay heat transient after
    the system is shutdown.  Furthermore, if the radiation dose at a
    critical point is too high, either for the other components of the
    system (\textsl{e.g.} fusion magnets) or for the personnel who
    must work with the system, the doses must be mitigated by the
    addition of shielding to the design.  Amongst many other factors,
    the cost of a system is affected by whether ``hands-on''
    maintenance is possible or remote handling is required.  When the
    lifetime of the system has been reached, the decommissioning cost
    is influenced by the levels of radioactivity in the various
    materials.
  
    The mathematical description of the activation process is quite
    straightforward.  The production rate of one isotope, $i$, from
    another, $j$, is proportional to the concentration of that other
    isotope, $N_j$.  The constant of proportionality is some reaction
    rate, $P_{j\rightarrow i}$, based on nuclear data.  For decay
    reaction paths, the production rate is equal to the product of the
    decay rate of isotope $j$ and the branching ratio for the reaction
    leading to $i$: $P_{j\rightarrow i} = \lambda_j b_{j\rightarrow
      i}.$ For nuclear reactions, the reaction rate is the inner
    product of the reaction cross-section for the reaction from $j$ to
    $i$ and the neutron flux, $P_{j\rightarrow i} = \int_0^\infty
    \sigma^{j\rightarrow i}(E)\phi(E) dE$.  With the cross-sections
    represented as group constants, as is usual, this production rate
    is: $P_{j\rightarrow i} = \sum_{g=1}^G \sigma_g^{j\rightarrow i}
    \phi_g.$ While each such term represents a production path for
    isotope $i$, it also represents one for the destruction paths for
    isotope $j$.  Assuming that there are no other sources (such as
    mass flux into the system), the rate of change of an isotope's
    concentration is simply the sum of all the production terms from
    other isotopes, $j$, and the destruction terms to other isotopes,
    $k$:
    \begin{equation*}
      \dot{N}_i(t) = \sum_{j=1}^n P_{j\rightarrow
        i}\left[\phi(t)\right]N_j(t) - \sum_{k=1}^n P_{i \rightarrow
        k}\left[\phi(t)\right]N_i(t).
    \end{equation*}
    By combining all the destruction rates to isotopes $k$ into a
    total destruction rate for isotope $i$, and writing the production
    rates as $P_{ij}$ without explicitly including the dependence on
    the neutron flux, this ordinary differential equation [ODE] is
    reduced to
    \begin{equation*}
      \dot{N}_i(t) = \sum_{j=1}^n P_{ij}N_j(t) - d_i N_i(t).
    \end{equation*}
    With one such equation for each of the isotopes in the activation
    tree, and one activation tree for each of the isotopes in the
    initial material, the large system of these coupled ordinary
    differential equations [ODE's] can be written in a matrix
    formulation as
    \begin{equation}
      \dot{\vec{N}}(t) = \mat{A}\vec{N}(t),\label{eqn:intro.basic_ode}
    \end{equation}
    where $A_{ii} = -d_i$ and $A_{ij} = P_{ij}$.  The formal solution
    to this equation is the matrix exponential:
    \begin{equation}
      \vec{N}(t) = e^{\mat{A}t}\vec{N}(0).\label{eqn:intro.basic_soln}
    \end{equation}
    
    For large problems with many initial isotopes, many fluxes at
    different spatial points, and complicated irradiation histories,
    an activation code is required to calculate the induced
    radioactivity levels.  An activation code must perform two
    distinct tasks: the physical modeling of the activation tree and
    the solution of the corresponding mathematical problem.  Starting
    with a list of initial isotopes, the code must first build the
    activation trees, deciding how large they need to be in order to
    include all the important contributions.  The trees are then
    converted into their mathematical equivalent and some technique is
    used to solve the matrix exponential problem.  Although these two
    tasks will be treated as distinct in this work, the way that the
    trees are built and sub-divided has an important impact on the
    kind of mathematical method that can be accurately implemented,
    and vice versa.
    
  \end{section}
  
  \begin{section}{Historical Overview}
    
    The computational solutions to this problem have been well
    studied\cite{adjoint}.  Many different approaches for modeling the
    physical problem have been combined with at least as many
    mathematical solution techniques.  Each combination has advantages
    and disadvantages, but none has arrived at an optimum mixture of
    accuracy, efficiency and usability.  Even ignoring the issue of
    usability (where this author feels many codes fail), there are few
    codes which are keeping up with the demands for greater accuracy
    in the physical models and mathematical solutions without becoming
    inconveniently slow.
  
    The predecessors to many modern activation codes were inventory
    codes (also called burn-up or depletion codes) designed for
    modeling the burn-up of nuclear fuel and build-up of fission
    products in nuclear reactors.  These codes use a variety of
    methods for modeling the physical system and solving the
    mathematical problem, but have historically been divided into
    three classes based on the mathematical method: time-step based
    ODE solvers, matrix exponential methods, and linear chain methods.
  
    The time-step based ODE solvers, such as used in
    FISPIN\cite{FISPIN}, use some algebraic approximation of the
    derivative on the left hand side of equation
    \ref{eqn:intro.basic_ode}.  One simple form of this approximation
    is based on the first principles definition of the derivative:
    \begin{equation*}
      \begin{split}
        \dot{N}(t) = \lim_{t_i - t_{i-1} \rightarrow 0} \frac{N(t_i)
          - N(t_{i-1})}{t_i - t_{i-1}}\\
        \therefore \dot{N}(t) \approx \frac{N(t_i)
          - N(t_{i-1})}{t_i - t_{i-1}}.
      \end{split}
    \end{equation*}
    For the activation problem, where $\vec{N}_i$ is the number
    density vector at time $i$, and $\Delta t = t_i - t_{i-1}$, this
    can be implemented simply in the explicit form:
    \begin{equation}
      \label{eqn:intro.time-step}
      \vec{N}_i = \Delta t \mat{A}\vec{N}_{i-1} + \vec{N}_{i-1}.
    \end{equation}
    More complicated differencing schemes with more accuracy can be
    developed based on Taylor series expansions in one or two
    variables, such as the well known Runge Kutta method.
  
    In all cases, however, to ensure accuracy these methods must use
    time steps small enough that the number density of any single
    isotope does not change too much during the time step.  For a
    problem with very short-lived isotopes, this time step must be
    very short, requiring many steps to solve the entire irradiation
    history, and therefore these methods can be very slow.
  
    The original matrix methods employed in inventory codes such as
    ORIGEN\cite{ORIGEN} calculate the series expansion of the
    $e^{\mat{A}t}$ exponential:
    \begin{equation*}
      e^{\mat{A}t} = \mat{I} + \mat{A}t + \frac{\mat{A}^2t^2}{2!} +
      \frac{\mat{A}^3t^3}{3!} + \ldots
    \end{equation*}
    In addition to being prone to round-off error, this expansion may
    need many terms to converge (if it does converge), which is
    computationally expensive, due to the large number of matrix
    multiplications.  More recently, this class of methods includes
    matrix decomposition methods to solve the matrix exponential (see
    chapter \ref{chap:math}).
  
    The final class of solution methods is based on a principle known
    as linear chains, of which CINDER\cite{CINDER} was one of the
    first implementations.  While both the time-step methods and the
    matrix exponential methods have traditionally attempted to solve
    the entire physical problem as one large system of ODE's, the
    linear chain method breaks the activation tree into a number of
    chains so that each isotope has a single production term and a
    single destruction term.  This creates a smaller system of ODE's
    in which the transfer matrix, \mat{A}, is exactly bidiagonal
    allowing an analytical solution commonly known as the Bateman
    equations\cite{Bateman} to be used:
    \begin{equation}
      \label{intro.bateman}
      N_i(t) = N_{i_o}e^{-d_i t} + \sum_{j=1}^{i-1}N_{j_o}\left [
        \sum_{k=j}^{i-1}\frac{P_{k+1}(e^{-d_k t} - e^{-d_i t})}{d_i -
          d_k}\prod_{\substack{l=j\\l\neq k}}^{i-1}\frac{P_{l+1}}{d_l-d_k}\right] \, ,
    \end{equation}
    where $P_{k+1} = P_{k \rightarrow k+1}$.
  
    One of the biggest limitations of this class of methods is its
    inability to model loops in the activation tree.  It can be seen
    in the above equation that there is a singularity when two of the
    destruction rates are identical.  While this may happen
    coincidentally in any linear chain, it is guaranteed to be the
    case if the same isotope occurs more than once in the chain.  This
    issue may not be significant for the simulation of fission reactor
    problems because most of the nuclear reactions required for loops
    (\textsl{e.g.} (n,p) or (n,2n) ) are threshold reactions with low
    or zero cross-sections in the energy domain of fission neutrons.
  
    Activation codes, many of which have been developed for fusion
    applications, also exist in each of these three classes of
    solution method.  Some are derived directly from an inventory code
    while others have no such obvious ancestry.  Conceptually
    activation calculations and inventory calculations are one and the
    same, but the wide variety in the nature of the systems being
    simulated in an activation problem can require more flexibility
    that an inventory code may provide.  For example, as suggested
    above, there are some reaction channels which are not relevant to
    fission inventory calculations.  In systems with higher energy
    neutron fluxes, such as fusion reactors or accelerator-based
    neutron sources, these additional channels can become important,
    if not dominant.  Additionally, the irradiation history for an
    activation calculation may be very different from that of a
    fission reaction system, with, for example, many frequent pulses.
  
    The most widely used fusion activation codes include
    FISPACT\cite{FISPACT}, a direct descendant of the FISPIN inventory
    code, RACC\cite{RACC} (and variations\cite{RACCP,RACCUW}),
    REAC\cite{REAC}, ACAB\cite{ACAB}, and DKR\cite{DKR} (and
    variations\cite{adjoint,DKRICF,DKRP}).  Each uses a different
    combination of physical and mathematical methods, enabling each to
    implement unique features.  For example, while FISPACT is limited
    to solutions at a single spatial point and cannot model pulsed
    histories exactly, the newest versions support sensitivity
    analyses and secondary activation caused by the light ions emitted
    by neutron reactions.  Newer versions of RACC switched from the
    time-step based GEAR ODE solver to matrix decomposition methods,
    enabling the efficient solution of pulsing histories with loops in
    the activation tree.  RACC's method for truncating the activation
    trees, however, is not ideal and may lead to an inaccurate final
    solution.  Not only was DKR the first to implement exact pulsing
    solutions, it is able to efficiently solve the activation problem
    across many points of a complicated geometry.  Due to its reliance
    on linear chains and the Bateman solution, however, DKR has long
    been criticized for its inability to handle loops in the
    activation tree.  Additionally, DKR has not implemented the
    ability to model the accumulation of light ions emitted from
    nuclear reactions.
  
    Despite the wide variety of available activation codes and their
    various capabilities, there is no single code which provides a
    complete range of these capabilities.  Furthermore, because of
    their original choice of physical methods, mathematical techniques
    and/or computational design, few, if any are extensible to include
    additional capabilities.
  
  \end{section}
  
  \begin{section}{Goals}
    
    The goal of this work is to design a fast, accurate and flexible
    activation code with a wide array of capabilities and features.
    In addition to establishing a base set of capabilities and
    features, it is important to look forward to a more advanced set
    and ensure that the new code will accomodate those features.
    \ALARA\ is designed to implement the following basic features:

    \renewcommand{\baselinestretch}{1.24}\normalsize
    \begin{description}
    \item[Problem Geometry]\ \\
      \vspace*{-1cm}
      \begin{itemize}
      \item simultaneous solution of activation problem at arbitrary
        number of spatial points
      \end{itemize}
    \item[Truncation]\ \\
      \vspace*{-1cm}
      \begin{itemize}
      \item user-defined calculation precision
      \end{itemize}
    \end{description}
    \newpage
    \begin{description}
    \item[Irradiation History]\ \\
      \vspace*{-1cm}
      \begin{itemize}
      \item multi-level pulsed operation histories
      \end{itemize}
    \item[Physical Modeling]\ \\
      \vspace*{-1cm}
      \begin{itemize}
      \item accurate handling of loops in the activation tree
      \item modeling of light ion accumulation
      \end{itemize}
    \item[User Features]\ \\
      \vspace*{-1cm}
      \begin{itemize}
      \item user-friendly input file format
      \item flexible geometry definition options
      \item user-defined output resolution
      \item user-defined output responses
      \end{itemize}
    \end{description}
    
    \noindent The advanced features of \ALARA\ include:
    \begin{description}
    \item[Physical Modeling]\ \\
      \vspace*{-1cm}
      \begin{itemize}
      \item modeling of reverse problem for detailed studies
      \end{itemize}
    \item[Irradiation History]\ \\
      \vspace*{-1cm}
      \begin{itemize}
      \item fully arbitrary operation schedules
      \end{itemize}
    \item[Mathematicl Techniques]\ \\
      \vspace*{-1cm}
      \begin{itemize}
      \item adaptive selection of mathematical method to optimize
        speed and accuracy
      \end{itemize}
    \end{description}
    \renewcommand{\baselinestretch}{1.66}\normalsize
    
    Prior to implementing these features, a careful analysis of the
    various methods for modeling the physical system, the techniques
    for solving the mathematical problem, and the way that they
    influence each other, must be performed.  Once the best methods and
    techniques have been determined, the importance of efficient and
    portable implementation must not be underestimated.  Even the best
    methods can be implemented poorly leading to inaccurate and slow
    solutions.
    
    \ALARA\ has been designed with three basic principles in mind:
    accuracy, speed, and simplicity.  These three qualities have been
    maximized in \ALARA\ after extensive research of the models
    involved in such calculations.  The errors, time of execution, and
    learning curve have all been made ``as low as reasonably
    achievable''.\footnote{This phrase is the origin of the term
      \ALARA, a well known philosophy in the nuclear industry related
      to the minimization of radiation exposure when working in
      radioactive environments.}  The methods used to model the
    physical system and to perform the mathematical solution are
    carefully combined to preserve or enhance the accuracy while
    accelerating the solution.  Throughout all this, there is an
    underlying effort to ensure that \ALARA\ be user-friendly by
    providing a simple, well-documented input file format, checking
    this input for errors, and providing a broad, flexible range of
    options.
    
    \begin{subsection}{Accuracy\label{ssec:intro.goals.acc}}
    
      Despite the list of shortcomings for existing codes, various
      studies\cite{IAEA.bench2.rep,ITER.exp.valid} have
      demonstrated that most of these codes achieve a reasonable
      degree of accuracy, compared to each other as well as compared
      to analytical solutions.  It is important, therefore, that
      \ALARA\ at least maintain this level of accuracy as it expands
      its range of modeling options.
  
      The accuracy of the final solution is affected both by how
      realistically the physical system is modeled and by what
      mathematical methods are employed for the final solution.
      Unfortunately, these two requirements often conflict; as the
      physical model becomes more realistic the required mathematical
      methods become more approximate or error prone.  When modeling
      the physical problem, two of the most important issues are how
      to deal with loops in the reaction scheme and how to truncate
      the theoretically infinite isotopic composition to a finite
      problem.  While the effect of the latter on the mathematical
      method is negligible, the former has a great impact.  In the
      past, the unwritten rule has been that realistic treatment of
      loops requires complicated/inefficient mathematical methods.
      \ALARA\ has broken that rule by finding a physical approximation
      to the loops which retains problem accuracy and allows for quite
      simple and efficient mathematical methods.  The keys to \ALARA's
      mathematical accuracy are its ability to adaptively choose the
      mathematical technique and the accuracy of those techniques.
      Two of the three mathematical techniques which \ALARA\ employs
      are mathematically exact!

    \end{subsection}
    
    \begin{subsection}{Speed\label{ssec:intro.goals.speed}}
      
      The most significant factor affecting the speed is the chosen
      class of mathematical method.  In particular, unless a linear
      transformation matrix method is used, the exact modeling of a
      pulsed history will require a long time.  \ALARA\ employs such
      matrix methods, solving for the linear transformation from the
      initial isotopic composition to the final composition for each
      pulse and inter-pulse dwell period, and then multiplying these
      matrices to obtain a complete linear transformation for the
      entire history.  In addition to this decision, speed was
      considered throughout the code design process.  For example,
      data library formats and internal data handling have been
      implemented with modern techniques to enhance versatility
      without sacrificing speed.

    \end{subsection}

    \begin{subsection}{Simplicity\label{ssec:intro.goals.simp}}
      
      While accuracy and speed have long been issues in the creation
      of engineering codes, their simplicity is of increasing
      importance.  In this context, simplicity is an issue for both
      modification/maintenance and use of the code.  Since \ALARA\ is
      written in $C\!\!\!\stackrel{+\!\!+}{}$, it benefits from some
      of the philosophies of object-oriented code design.  This allows
      the code itself to be more readable to future programmers and
      facilitates enhanced modularity.  This modularity means that if
      new functionality is added to the code, it can be optimized
      internally with minimal detrimental effect on the existing code.
  
      \ALARA\ has been designed with the user in mind.  Even though
      improved methods have existed for years, many codes have
      continued to use input formats which are reminiscent of punch
      card input entry.  Furthermore, most tools in this field have
      been designed for the solution at a single spatial point,
      requiring many subsequent and slightly altered runs to get any
      kind of spatial information.  \ALARA\ allows the user to find
      the solution to an activation problem in a variety of different
      multi-dimensional geometries, using a flexible system to define
      the material properties and allowing a complicated
      pulsed/intermittent irradiation history and a variety of
      after-shutdown solution times.  Furthermore, the input file can
      be fully commented, preventing the common difficulty of creating
      a long list of seemingly disconnected numbers for code input.
  
    \end{subsection}


  \end{section}
  
  
\end{chapter}
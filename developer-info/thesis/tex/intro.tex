\begin{chapter}{Introduction and Background}

\begin{section}{Problem Definition\label{sec:intro.prob_def}}

  \begin{floatingfigure}{0.55\columnwidth}
    \begin{center}
      \epsfig{file=eps/basic_tree.eps,width=0.5\columnwidth}
      \caption{A sample activation tree showing the results of
        activation of isotope A.}\label{fig:intro.basic_tree}
    \end{center}
  \end{floatingfigure}
  
  Activation is one of the many responses induced in materials when
  irradiated with neutrons.  When an isotope is subjected to neutron
  irradiation, it is possible that the neutrons will interact with the
  nuclei of that isotope, converting them to different isotopes.  Many
  such reactions are possible with each isotope, so that after only
  one round of neutron reactions, a material made of only one isotope
  can be partially converted into over 20 others.  These isotopes, in
  turn, can undergo similar interactions, leading to yet more
  isotopes, and so on.  Many of these isotopes can and will be
  radioactive, and through their decay, even more isotopes can enter
  the physical system.  If represented graphically (Figure
  \ref{fig:intro.basic_tree}), this process forms a tree of isotopes,
  where each of the branch in the tree represents either a nuclear
  reaction or a nuclear decay.
    
  The end result of this so called activation process is a radioactive
  material.  After calculating the concentrations of all the various
  isotopes created by the activation process, other engineering
  responses can be determined.  Using the half-lives of the
  radioactive isotopes, the radioactivity of the material can be
  determined.  Given the average energy of each decay, these
  radioactivity values can be converted to decay heat results.  Once
  the radioactivity is known, it can be compared to regulated limits
  to determine how the material must be handled.  If the radioactive
  isotopes emit gamma rays, the radioactivity results can be
  translated into sources of gamma rays, which may be used to
  calculate a radiation dose at some point.  These radiation doses
  determine, in turn, whether systems and components can be maintained
  by hand or require robotic systems and remote handling.
    
  All of these responses are very important when designing, operating
  and costing a nuclear system.  For safety considerations, it is
  important to know the inventory of all the radioactive isotopes
  which may be released and to know the decay heat transient after the
  system is shutdown.  Furthermore, if the radiation dose at a
  critical point it too high, either for the other components of the
  system or for the personnel which must work on the system, addition
  shielding must be added to the design to mitigate this.  Amongst
  many other factors, the cost of a system is affected by whether the
  maintenance can be done ``hands-on'' or remoted handling is
  required.  When the lifetime of the system has been reached, the
  cost of decommissioning is influenced by the levels of radioactivity
  in the various materials.
  
  The mathematical description of the activation process is quite
  straightforward.  The production rate of one isotope, $i$, from
  another, $j$, is proportional to the concentration of that other
  isotope, $N_j$.  The constant of proportionality is some reaction
  rate, $P_{j\rightarrow i}$, based on nuclear data and possibly
  dependent on the neutron flux, $\phi(t)$.  Logically, this same term
  represents part of the destruction rate of isotope $j$.  Assuming
  that there are no other sources for any (such as new material being
  added to the system), the rate of change of an isotope's
  concentration is simply the sum of all the production terms from
  other isotopes, $j$, and the destruction terms to other isotopes,
  $k$:
  \begin{equation*}
    \dot{N}_i(t) = \sum_{j=1}^n P_{j\rightarrow
      i}\left[\phi(t)\right]N_j(t) - \sum_{k=1}^n P_{i \rightarrow
      k}\left[\phi(t)\right]N_i(t).
  \end{equation*}
  By combining all the production
  rates of isotopes $k$ into a total destruction rate for isotope $i$,
  and not explicitly including the dependence on the neutron flux,
  this ordinary differential equation [ODE] is reduced to
  \begin{equation*}
    \dot{N}_i(t) = \sum_{j=1}^n P_{j\rightarrow i}N_j(t) - d_i
    N_i(t).
  \end{equation*}
  With one such equation for each of the isotopes in the activation
  tree, and one activation tree for each of the isotopes in the
  initial material, the large system of these coupled ordinary
  differential equations [ODE's] can be written in a matrix
  formulation as
  \begin{equation}
    \dot{\vec{N}}(t) = \mat{A}\vec{N}(t),\label{eqn:intro.basic_ode}
  \end{equation}
  where the term $-d_i$
  appears on the diagonal in row $i$ and the term $P_{j \rightarrow
    i}$ appears in column $j$ of row $i$.  The formal solution to this
  equation is the matrix exponential:
  \begin{equation}
    \vec{N}(t) = e^{\mat{A}t}\vec{N}(0).\label{eqn:intro.basic_soln}
  \end{equation}

  For larger problems, with many initial isotopes, many fluxes at
  different spatial points, and complicated irradiation histories, an
  activation code is required to calculate the levels of induced
  radioactivity.  The tasks for an activation calculation can be
  divided into two distinct areas: physical modelling and mathematical
  solution.  First the code must build an activation tree, deciding
  how large it should be in order to include all the important
  contributions.  The tree is converted into its mathematical
  equivalent and then some method must be used to solve the matrix
  exponential problem.  Although these two tasks will be treated as
  distinct in this work, the way that the trees are built and
  sub-divided has an important impact on the kind of mathematical
  method that can be accurately implemented.

\end{section}

\begin{section}{Historical Overview}
  
  The computational solutions to this problem have been well studied.
  Many different approaches for modeling the physical problem have
  been combined with at least as many mathematical solution
  methodologies.  Each combination has its advantages and
  disadvantages, but none have arrived at an optimum mixture of
  accuracy, efficiency and usability.  Even ignoring the issue of
  usability (where this author feels most codes all fail), there are
  few codes which are keeping up with the demands of greater accuracy
  in modeling and solutions without becoming inconveniently slow.

  The predecessors to many modern activation codes were inventory
  codes designed for modelling the burn-up of nuclear fuel and
  build-up of fission products in nuclear reactors.  These codes used
  a variety of methods for modelling the physical system and solving
  the mathematical problem, but have historically been divided into
  three classes based on the mathematical method: time-step based ODE
  solvers, matrix exponential methods, and linear chain methods.
  
  The time-step based ODE solvers, such as used in FISPIN, use some
  algebraic approximation of the derivative on the left hand side of
  equation \ref{eqn:intro.basic_ode}.  One simple form of this
  approximation uses is based on the first principles definition of
  the derivative:
  \begin{equation*}
    \begin{split}
      \dot{N}(t) = \lim_{t_i - t_{i-1} \rightarrow 0} \frac{N(t_i)
        - N(t_{i-1})}{t_i - t_{i-1}}\\
      \therefore \dot{N}(t) \approx \frac{N(t_i)
        - N(t_{i-1})}{t_i - t_{i-1}}.
    \end{split}
  \end{equation*}
  For the activation problem, where $\vec{N}_i$ is the number density
  vector at time $i$, this can be implemented simple in the explicit
  form:
  \begin{equation}
    \label{eqn:intro.time-step}
    \vec{N}_i = \Delta t \mat{A}\vec{N}_{i-1} + \vec{N}_{i-1}.
  \end{equation}
  More complicated differencing schemes with more accuracy can be
  developed based on Taylor series expansions in one or two variables,
  such as the well known Runge Kutta method.
  
  In all cases, however, to ensure accuracy these methods must use
  time steps small enough that the number density of any single
  isotope does not change too much during the time step.  For a
  problem with very short-lived isotopes, this time step must be very
  short, requiring many steps to solve the entire irradiation
  history, and thus these methods can be very slow.
  
  The original matrix methods employed in inventory codes such as
  ORIGEN calculate the series expansion of the $e^{\mat{A}t}$
  exponential:
  \begin{equation*}
    e^{\mat{A}t} = \mat{I} + \mat{A}t + \frac{\mat{A}^2t^2}{2!} +
    \frac{\mat{A}^3t^3}{3!} + \ldots
  \end{equation*}
  In addition to being prone to round-off error, this expansion may
  need many terms to converge (if it does converge), which is
  computationally expensive, due to the large number of matrix
  multiplications,
  
  The final class of solution methods is linear chains, of which
  CINDER was one of the first.  While both the time-step methods and
  the matrix exponential methods have traditionally attempted to solve
  the entire problem at one time, using a single large system of
  ODE's, the linear chain method breaks the activation tree into a
  number of chains so that each isotope has a single production term
  and a single destruciton term.  This creates a smaller system of
  ODE's in which the transfer matrix, \mat{A}, is exactly bidiagonal
  allowing an analytical solution commonly known as the Bateman
  equations to be used:
  \begin{equation}
    \label{intro.bateman}
    N_i(t) = N_{i_o}e^{-d_i t} + \sum_{j=1}^{i-1}N_{j_o}\left [
      \sum_{k=j}^{i-1}\frac{P_{k+1}(e^{-d_k t} - e^{-d_i t})}{d_i -
        d_k}\prod_{\substack{l=j\\l\neq k}}^{i-1}\frac{P_{l+1}}{d_l-d_k}\right] \, .
  \end{equation}
  
  One of the biggest limitations of this class of methods is its
  inability to model loops in the activation tree.  It can be seen in
  the above equation that there is a singularity when two of the
  destruction rates are identical.  While this may happen
  coincidentally in any linear chain, it is guaranteed to be the case
  if the same isotope occurs more than once in the chain.  This issue
  is not significant for the simulation of a fission reactor because
  most of the nuclear reactions which are required for loops
  (\textsl{e.g.} (n,p) or (n,2n) ) are threshold reactions with low or
  zero cross-section in the energy domain of fission neutrons.
  
  Activation codes, many of which have been developed for fusion
  applications, also exist in each of these three classes of solution
  method.  Some are derived directly from an inventory code while
  others have no such obvious ancestry.  Conceptually activation
  calculations and inventory calculations are one and the same, but
  the wide variety in the nature of the system being simulated in an
  activation problem can require more flexibility that an inventory
  code may provide.  For example, as suggested above, there are some
  reaction channels which are not relevant to fission inventory
  calculations.  In systems with higher energy neutron fluxes, such as
  fusion reactors or accelerator-based neutron sources, these
  additional channels can become important, if not dominant.
  Additionally, the irradiation history for an activation calculation
  may be very different from that of a fission reaction system, with,
  for example, many frequent pulses.
  
  In the time-step based ODE solver class of codes, there are many
  codes which have seen widespread use for fusion activation,
  including FISPACT, a direct descendent of FISPIN, and RACC.  Used as
  the main European fusion activation code, FISPACT, is based on the
  standard Euler method for solving ODE's.  
  
  
  
\end{section}

\begin{section}{Goals}
  
\end{section}


\end{chapter}
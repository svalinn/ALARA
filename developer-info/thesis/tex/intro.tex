\begin{chapter}{Introduction and Background}

\begin{section}{Problem Definition\label{sec:intro.prob_def}}

  \begin{subsection}{What is Activation?\label{sec:intro.prob_def.act}}
    
    Activation is one of the many responses induced in materials when
    irradiated with neutrons.  When an isotope is subjected to neutron
    irradiation, it is possible that a neutron will interact with a
    nucleus of that isotope, converting it to a different isotope.
    Many such reactions are possible with each isotope, so that after
    only one round of neutron reactions, a material made of only one
    isotope can be partially converted into over 20 others.  These
    isotopes, in turn, can undergo similar interactions, leading to
    yet more isotopes, and so on.  Many of these isotopes can and will
    be radioactive, and through their decay, even more isotopes can
    enter the physical system.  The end result of this neutron
    irradiation is a radioactive material, and the process is called
    activation.  If represented graphically (Figure
    \ref{fig:intro.basic_tree}), this process forms a tree of
    isotopes, where each of the branch in the tree represents either a
    nuclear reaction or a nuclear decay.
    
  \end{subsection}

  \begin{subsection}{Role of Activation in System/Reactor Design?}
    
    Activation plays an important role in the design, operation and
    cost of any system exposed to a significant neutron flux, such as
    accelerator-based neutron sources of fusion reactors.  
  
  \end{subsection}
  
  \begin{subsection}{Problem to be solved by Activation Code}
    
  \end{subsection}
\end{section}

\begin{section}{Historical Overview}
  
\end{section}

\begin{section}{Goals}
  
\end{section}


INCLUDE SOMETHING TO THIS EFFECT:

Even though the choice physical and mathematical models influence each
other closely, they will be treated as separable here, with comments
about their influence as necessary.

\end{chapter}
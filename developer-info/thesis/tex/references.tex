\begin{thebibliography}{100}
\addcontentsline{toc}{chapter}{\bibname}


\bibitem{adjoint} White, A., ``Transmutation and Activation
  Analysis of Fusion Power Plants,'' Doctoral Thesis, University of
  Wisconsin-Madison, (May 1985).
  
\bibitem{FISPIN} Barstall, R.F., ``FISPIN - A Computer Code for
  Nuclide Inventory Calculations,'' ND-R-328(R), (October 1979).
  
\bibitem{ORIGEN} Bell, M., ``ORIGEN--The ORNL Isotope Generation and
  Depletion Code,'' Oak Ridge National Laboratory report ORNL-4628,
  (May 1973).
  
\bibitem{CINDER} England, T.R., ``CINDER - A One-Point Depletion and
  Fission Product Program,'' Bettis Atomic Power Laboratory report
  WAPD-TM-334(Rev), (1964).
  
\bibitem{Bateman} Bateman, H., \textsl{Proc. Cambridge Phil. Soc.}
  \textbf{15} 423 (1910).
  
\bibitem{FISPACT} Forrest, R.A., and J-Ch.~Sublet, ``FISPACT3 - User
  Manual,'' AEA/FUS 227, (April 1993).
  
\bibitem{RACC} Jung, J., ``RACC: Theory and Use of the Radioactivity
  Code RACC,'' Argonne National Laboratory report ANL/FPP/TM-122, (May
  1979).
  
\bibitem{RACCP} Attaya, H., ``Input Instructions for RACC-P,'' Argonne
  National Laboratory report ANL/FPP/TM-270, (September 1994).
  
\bibitem{RACCUW} Wang, Q., and D.L.~Henderson, ``Summary Report for
  ITER Design Task D10: Updating the Activation Code RACC for ITER
  Design Analysis,'' University of Wisconsin Fusion Technology
  Institute report UWFDM-977, (1995).
  
\bibitem{REAC} Mann, F.M., ``REAC*2: Users Manual and Code
  Description,'' Westinghouse Hanford Company report WHC-EP-0282,
  (December 1989).
  
\bibitem{ACAB} Sanz, J., \textsl{et al}, ``ACAB, Activation Code for
  Fusion Applications: User's Manual V 2.0,'' Lawrence Livermore
  National Laboratory report UCRL-MA-122002, (September 1995).
  
\bibitem{DKR} Sung, T.Y., and W.F.~Vogelsang, ``DKR: A Radioactivity
  Calculation Code for Fusion Reactors,'' University of Wisconsin
  Fusion Technology Institute report UWFDM-170, (September 1976).
  
\bibitem{DKRICF} Henderson, D.L., and O.~Yasar, ``DKRICF: A
  Radioactivity and Dose Rate Calculation Code Package: Vols. I \&
  II,'' University of Wisconsin Fusion Technology Institute report
  UWFDM-714, (November 1986).  This code
  package is available from the Radiation Shielding Information Center
  (RSIC) at Oak Ridge National Laboratory as Computer Code Collection
  entry CCC-323-DKR.
  
\bibitem{DKRP} DKR-PULSAR is a new version of the DKR-ICF code which
  implements methods from Reference \citen{Pulsar} for the exact
  treatment of pulsed history irradiation.  It is being developed by
  D.L.~Henderson and H.~Khater at the University of Wisconsin--Madison.
  
\bibitem{ITER.exp.valid} Taylor, N., \textsl{et al.}, ``Experimental
  validation of calculations of decay heat induced by 14 MeV neutron
  activation of ITER materials'', \textsl{Fus. Eng. and Design},
  \textbf{45}, (March 1999).
  
\bibitem{IAEA.bench2.rep} Cheng, E.T., R.A.~Forrest, and
  A.B.~Pashchenko, ``Report on the Second International Activation
  Calculation Benchmark Comparison Study,'' International Atomic
  Energy Agency report INDC(NDS)-300, (February 1994).
  
\bibitem{UWFDM} Wilson, P.P.H., and D.L.~Henderson, ``Expanding
  Towards Excellence: Ironing out DKR's Wrinkles,'' University of
  Wisconsin Fusion Technology Institute report UWFDM-995, (1995).
  
\bibitem{EAF} Sublet, J.-Ch., J.~Kopecky and R.A.~Forrest, ``The
  European Activation File, EAF-97 - Cross section library,'' United
  Kingdom Atomic Energy Agency Fusion report UKAEA-FUS-351, (June
  1997).
  
\bibitem{FENDL2} Pashchenko, A.B., \textsl{et al.}, ``FENDL/A-2.0
  Neutron activation cross section data library for fusion
  applications,'' International Atomic Energy Agency report
  IAEA(NDS)-173 (October 1998).  Data library retrieved online from
  the IAEA Nuclear Data Section.

\bibitem{Mann}Mann, F.M., and D.E.~Lessor, ``REAC*3 Nuclear Data
  Libraries,'' Proceedings of an International Conference on Nuclear
  Data entitled: \underline{Nuclear Data for Science and Technology},
  held at the Forschungszentrum Juelich, Fed. Rep. of Germany, 13-17
  May 1991.

\bibitem{Pulsar}Sisolak, J.E., S.E.~Spangler, and D.L.~Henderson,
  ``Pulsed/Intermittent Activation in Fusion Energy Reactor Systems,''
  \textsl{Fusion Tech.} \textbf{21}, 2145 (1992).
  
\bibitem{spanglerMS} Spangler, S.E., ``A Numerical Method for
  Calculating Nuclide Densities in Pulsed Activation Studies,'' Master
  of Science Thesis, University of Wisconsin-Madison, (August 1991).
  
\bibitem{spangler} Spangler, S.E., J.E.~Sisolak, and D.L.~Henderson,
  ``Calculational Models for the Treatment of Pulsed/Intermittent
  Activation Within Fusion Energy Devices'' \textsl{Fus. Eng. and
    Design}, \textbf{22}, 349 (July 1993).
  
\bibitem{UCBerkeley.NIF.Target} de Hoon, M.L.J., E.~Greenspan and
  M.D.~Lowenthal, ``A Model for Pulsed Activation Accounting for
  Circulation, Extraction and Makeup,'' Abstracts of the Thirteenth
  American Nuclear Society Topical Meeting on the Technology of Fusion
  Energy, Nashville, TN, (June 1998).
  
\bibitem{UKA.Thesis} Wilson, P.P.H., ``Neutronics of the IFMIF Neutron
  Source: Development and Analysis,'' Forschungszentrum Karlsruhe
  report FZKA-6218 (1999).
  
\bibitem{DUBIOUS} Moler, C. and C.~Van Loan, ``Nineteen Dubious Ways to
  Compute the Exponential of a Matrix,'' \textsl{SIAM
    Review}, \textbf{20}, 801 (Oct.~1978).
  
\bibitem{eigenvectors} Fukumoto, H., ``New Approach to
  Neutron-Induced Transmutation, Radioactivity and Afterheat
  Calculations and Its Application to Fusion Reactors,'' \textsl{Nuc. Sci.
  and Tech.}, \textbf{23}, 97, (February 1986).
  
\bibitem{GERAPH} Wilson, P.P.H., J.E.~Sisolak, and D.L.~Henderson,
  ``GERAPH: A Novel Approach to the General Solution of Pulsed History
  Activation Problems,'' \textsl{Fusion Tech.} \textbf{26}, 1092
  (November 1994).
  
\bibitem{UofT_csc280} Lee, E.S., ``Computer Engineering: Computer
  Algorithms, Data Structures, and Languages,'' Prepared Notes,
  University of Toronto, (1989).
  
\bibitem{IAEA.bench1.spec} Sawan, M.E., ``FENDL Activation Benchmark:
  Specifications for the Calculational Activation Benchmark,''
  International Atomic Energy Agency report INDC(NDS)-318, (October
  1994).
  
\bibitem{ONEDANT} O'Dell, R.D., \textsl{et al.}, ``User's Manual for
  ONEDANT - A Code Package for One-Dimensional, Diffusion-Accelerated,
  Neutral-Particle Transport,'' Los Alamos National Laboratory report
  LA-9184-M, Rev., (February 1989).
  
\bibitem{hosny} Attaya, H., ``Radioactivity Computation of Steady
  State and Pulsed Fusion Reactor Operation,'' \textsl{Fusion Eng. and
  Design}, \textbf{28} 571 (1995).

\bibitem{qingming} Wang, Q., and D.L.~Henderson, ``Pulsed Activation
  Analyses of the ITER Blanket Design Options Considered in the
  Blanket Trade-off Study,'' \textsl{Fusion Eng. and Design},
  \textbf{28} 579 (1995).
  
\bibitem{sequential} Cierjacks, S.W., et al, ``Development of a Novel
  Algorithm and Production of New Nuclear Data Libraries for the
  Treatment of Sequential $(x,n)$ Reactions in Fusion Material
  Activation Calculations,'' \textsl{Fus. Tech.}, \textbf{24}, 277
  (November 1993)
    
\bibitem{sensPhD} Khursheed, A., ``Neutron-Induced Activation of
  Materials for the First Wall of Conceptual Fusion Reactors,''
  Doctoral Thesis, Imperial College of Science and Technology, London,
  England, (1989).
    
\bibitem{sensJames} James, M.F., ``The Calculation of Sensitivities
  of Nuclide Inventories and Decay Power,'' Proceedings of the NEA
  Specialist Meeting on Data for Decay Heat Predictions, Studsvik,
  Sweden, 7-10 September 1987.
    
  
\end{thebibliography}
